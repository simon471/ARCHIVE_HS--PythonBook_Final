\chapter{The Boolean}

\section{Introduction}

 Boolean is a data type used in computer, there having two value and usually is "True" and "False".
 And Intended to represent the truth values of logic and Boolean algebra.\\ \ \\ \ \\

\section{The Boolean Operators}

Boolean Operators are some words like (AND, OR, NOT) used as
connection to combine or remove words in a research , resulting in more focused and
productive results. This should save time and effort by eliminating inappropriate hits that
must be scanned before discarding, i thing these operators can greatly 
reduce or expand the amount of records returned.\\ \ \\

\noindent Using these operators can reduce or expand the amount of records returned.
Boolean operators are useful in saving time by focusing searches for more 'on-target'
And now i would like to introduce the operators one by one with some examples.\\ \ \\ \ \\

AND -- requires both terms to be in each item returned. If one term is contained in the
document and the other is not, the item is not included in the resulting list. (Simpifly the
search)\\ \ \\

\noindent Example:
\begin{verbatim}
True AND True = True
True AND False = False
False AND True = False
False AND False = False
\end{verbatim}\\ \ \\
\ \\ \ \\

OR -- either term (or both) will be in the returned document. (Broadens the search)\\ \ \\

\noindent Example:
\begin{verbatim}
True OR True = True
True OR False = True
False OR True = True
False OR False = False
\end{verbatim}\\ \ \\

NOT -- first term is searched, then any records containing the term after the operators are subtracted from the results.The result need to be the opposite.\\ \ \\

\noindent Example:
\begin{verbatim}
NOT(True) = False
NOT(False) = True
\end{verbatim}\\ \ \\
\ \\ \ \\ 

\section{Boolean Arithmetic}

We can get booleans by making some statements like calculation symbols, and now we have " != "," == ", " > ", " < ", " <= ", " >= ".\

\begin{framed}
\begin{verbatim}
" != " symbol is means not EQUAL.
" == " symbol is what we call equal.
" > " symbol means greater than.
" < " symbol means less than.
" >= " symbol means greater and equal to.
" <= " symbol means less and equal to.
\end{verbatim}\\ \ \\
\end{framed}

\noindent For example:
\begin{verbatim}
" == "
9==8 will get False.   
9==9 will get True.                
NOT(9==8) will get True.      
NOT(9==9) will get False.     

" > "
9>8 will get True.
8>9 will get False.
NOT(9>8) will get False.
NOT(8>9) will get True.

" <= "
9<=8 will get False.
8<=9 will get True.
8<=8 will get True.
NOT(9<=8) will get True.
NOT(8<=9) will get False.
NOT(8<=8) will get False.
\end{verbatim}

Now we have some example for combination with symbols.And .Lets try it together.\ \\ \

\begin{verbatim}

EXAMPLES:

NOT(9<7) AND (6>=7 OR 8==8)


(9*3>=35) OR NOT((77/11)==7) AND (6>=6 OR 54>4) AND (8<5 OR 33+3==10*4-4)


(9+2-4*2<=36/6) AND (8!=9 OR NOT(5*5-65/5==5)


NOT(6+4==7+3-5+4-1+24/8) OR ((6<3*3)!=(4+4+5-3)


\end{verbatim}\\ \ \\







