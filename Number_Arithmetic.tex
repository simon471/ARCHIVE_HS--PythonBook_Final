\chapter{Arithmetic}

\section{introduction}
 In any computer languages, number arithmetic is important. It is the basic of coding.
   
\section{Addition and Subtraction}
 + means add and - means minus. Which is simple. if you code in:
 
 2+3
      
 4-3\\

 you get:
 
 5
 
 1
 
 \section{Multiplication, Parentheses, and Precedence}
 Python use * as multiply and use ** as power. if you code in:
 
 3*4
 
 3*7
 
 3**2
 
 4**3\\
 
 you get:
 
 12
 
 21
 
 9
 
 64
 
Python uses the normal precedence of arithmetic operations: Multiplications and divisions are done before addition and subtraction, unless there are parentheses.  Code in:
 
 (2+3)*4
 
 2+3*4\\
 
 You get:
 
 20
 
 14
 
  \section{Division and Remainders}
  
  In the earliest grades you would say ?14 divided by 4 is 3 with a remainder of 2?. The problem here is that the answer is in two parts, the integer quotient 3 and the remainder 2, and neither of these results is the same as the decimal result.Python has separate operations to generate each part. Python uses single division symbol / for the operation that produces  the quotient with decimal and uses the the doubled division symbol // for the operation that produces just the integer quotient, and introduces the symbol \% for the operation of finding the remainder. If you code in:
  
  5/2
  
  5.2/2
  
  5.2//2
  
  5\%3
  
  6\%8\\
  
  you get:
  
  2
  
  2.6
  
  2.0
  
  2
  
  6
 

